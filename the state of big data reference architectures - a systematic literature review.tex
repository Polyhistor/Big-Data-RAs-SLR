\documentclass[review]{elsarticle}

\usepackage{lineno,hyperref}
\modulolinenumbers[5]

\journal{Journal of \LaTeX\ Templates}

%%%%%%%%%%%%%%%%%%%%%%%
%% Elsevier bibliography styles
%%%%%%%%%%%%%%%%%%%%%%%
%% To change the style, put a % in front of the second line of the current style and
%% remove the % from the second line of the style you would like to use.
%%%%%%%%%%%%%%%%%%%%%%%

%% Numbered
%\bibliographystyle{model1-num-names}

%% Numbered without titles
%\bibliographystyle{model1a-num-names}

%% Harvard
%\bibliographystyle{model2-names.bst}\biboptions{authoryear}

%% Vancouver numbered
%\usepackage{numcompress}\bibliographystyle{model3-num-names}

%% Vancouver name/year
%\usepackage{numcompress}\bibliographystyle{model4-names}\biboptions{authoryear}

%% APA style
%\bibliographystyle{model5-names}\biboptions{authoryear}

%% AMA style
%\usepackage{numcompress}\bibliographystyle{model6-num-names}

%% `Elsevier LaTeX' style
\bibliographystyle{elsarticle-num}
%%%%%%%%%%%%%%%%%%%%%%%

\begin{document}

\begin{frontmatter}

\title{The state of big data reference architectures: a systematic literature review}

%% Group authors per affiliation:
% \author{Elsevier\fnref{myfootnote}}
% \address{Radarweg 29, Amsterdam}
% \fntext[myfootnote]{Since 1880.}

%% or include affiliations in footnotes:
% \author[mymainaddress,mysecondaryaddress]{Elsevier Inc}
% \ead[url]{www.elsevier.com}

% \author[mysecondaryaddress]{Global Customer Service\corref{mycorrespondingauthor}}
% \cortext[mycorrespondingauthor]{Corresponding author}
% \ead{support@elsevier.com}

% \address[mymainaddress]{1600 John F Kennedy Boulevard, Philadelphia}
% \address[mysecondaryaddress]{360 Park Avenue South, New York}

\begin{abstract}
This template helps you to create a properly formatted \LaTeX\ manuscript.
\end{abstract}

\begin{keyword}
\texttt{elsarticle.cls}\sep \LaTeX\sep Elsevier \sep template
\MSC[2010] 00-01\sep  99-00 s
\end{keyword}

\end{frontmatter}

\linenumbers

\section{Introduction}

The rapid development of software technologies, the proliferation of digital devices and networking infrastructure of today, have by and large, augmented user’s capability to generate data \cite{AtaeiSecurity}. In the age of information, users are unceasing generators of structured, semi-structured, and unstructured data that if collected and crunched correctly, may reveal game-changing patterns \cite{AtaeiACIS}.

The unprecedented proliferation of data have emerged a new ecosystem of technologies; one of these ecosystems is big data (BD)\cite{AtaeiHype}. BD is a term emerged to describe large amount of data that comes in various forms from different channels. Within the years, BD has attained a lot of attention from academia and industry, and many strive to benefit from this new material. Howbeit, adopting BD requires the absorption of great deal of complexity and many traditional systems cannot cope with characteristics of this domain. 

A recent survey published by Databricks in partnership with MIT Technology Review Insights, stated that only 13\% of companies excel at delivering on their data strategy \cite{DataBricks}. In the same vein, Vintage Partners highlighted that only 24\% of companies have successfully adopted BD \cite{NewVantageSurvey}. Sigma computing report presented that 1 in 4 business experts have given up on getting insights they needed because the data processing took too long \cite{SigmaSurvey}. Moreover, Gartner approximated that only 20\% of companies have successfully adopted BD. 

Some of the most highlighted challenges of BD is 'lack of business context', 'organizational challenges', 'BD architecture', 'data engineering', 'rapid technology change', and 'lack of talent' \cite{AtaeiBigDataEnvirons}. Whereas similar issues may exist in other domains, it is exacerbated when it comes to BD systems. This is due the inherent complexity of BD engineering, the need for real-time processing, the scalability requirement of these systems, and the sensitivities around data.

Today, majority of BD systems are designed underlying ad-hoc and complicated architectural solutions \cite{Gorton}, that do not seem to adhere to similar patterns. This will challenge software architects to design a suitable solution for any given context, creates a foundation for an immature architectural decision, and does not promote the growth and development of BD systems as a whole. 

Therefore, since the approach of ad-hoc design to BD systems is undesirable and leaves many engineers in the dark, there is a need for more software engineering research for BD systems. To this end, this study presents a systematic literature review (SLR) on BD (BD) reference architectures (RAs). 

\section{Why reference architectures?}
Conceptualization of the system as an RA, helps with understanding of the system’s key components, behavior, composition and evolution of it, which in turn affect quality attributes such as maintainability, scalability and performance \cite{Cloutier}. Therefore RAs can be a good standardization artefact and a communication medium that not only results in concrete architectures for BD systems, but also provide stakeholders with unified elements and symbols to discuss and progress BD projects.

This approach to system development is not new to practitioners of complex system. In software product line (SPL) development, RAs are utilized as generic artifacts that are instantiated and configured for a particular domain of systems \cite{Derras}. In software engineering, IT giants like IBM have referred to RAs as the 'best of best practices' to address complex and unique system design challenges \cite{Cloutier}. In other international standardization, RAs have been repeatedly used to standardize an emerging domain, a good example of this is BS ISO/IEC 18384-1 RA for service oriented architectures \cite{Iso18384-1}. 

\section{Why this systematic review is necessary?}

Despite the undeniable benefits of RAs, and their potential to solve some of the complex issues of BD 

\begin{itemize}
\item document style
\item baselineskip
\item front matter
\item keywords and MSC codes
\item theorems, definitions and proofs
\item lables of enumerations
\item citation style and labeling.
\end{itemize}

\section{Front matter}

The author names and affiliations could be formatted in two ways:
\begin{enumerate}[(1)]
\item Group the authors per affiliation.
\item Use footnotes to indicate the affiliations.
\end{enumerate}
See the front matter of this document for examples. You are recommended to conform your choice to the journal you are submitting to.

\section{Bibliography styles}

There are various bibliography styles available. You can select the style of your choice in the preamble of this document. These styles are Elsevier styles based on standard styles like Harvard and Vancouver. Please use Bib\TeX\ to generate your bibliography and include DOIs whenever available.

Here are two sample references: \cite{Feynman1963118,Dirac1953888}.

\section{Improvements}

\begin{enumerate}
    \item The current writing style looks like a summary description, lacks new insight on the topic. The overall contribution needs to be enhanced.
    \item The author raises seven research questions, but how does the author develop these questions? Are these real questions that have never been discussed? The research questions need to be developed according to the literature. In this way, we can realize what is the gap on this topic.
    \item The inclusion criteria and exclusion criteria are ambiguous and questionable. Is it possible for readers to reproduce this study according to these criteria? Did the author perform the reliability and validity tests? The author needs to provide more detail about the review methodology.
    \item In general, each larger-scale system requires a more understanding of architectural components, owing largely to the complex nature of system architects. However, I cannot find a case that the authors demonstrate the uniqueness of BD systems, and the actual development challenges in BD systems.
    \item Although this study adapts SLR approach, it should not completely miss a literature reviewing section. It is necessary to provide to the reader the preliminary details which are necessary to understand the purpose of this study, techniques and key concerns of the various research work that the authors have reviewed. There is no theoretical argument to support the development of the research questions.
    \item The findings yielded by investigating the research questions of this SLR should constitute many discussion points around the research and practice of BD systems. However, the manuscript is completely missing a discussion section. One should expect that the results of SLR can inform the current knowledge and provide several research directions for future research. 
    \item Last, one of the core challenges with the paper is to situate it within an ongoing scholarly conversation. The authors currently reference a fairly diverse set of papers, but remain at a fairly abstract level when it comes to elaborating how your work builds upon and expands existing work. In turn, this makes it difficult to appreciate theoretical implications of your work. 
\end{enumerate}


\section*{References}

\bibliography{mybibfile}

\end{document}